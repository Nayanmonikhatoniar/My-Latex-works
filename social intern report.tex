\documentclass[a4paper,12pt]{article}

% Packages for styling and layout
\usepackage[utf8]{inputenc}
\usepackage{graphicx}
\usepackage{titlesec}
\usepackage{geometry}
\usepackage{color, xcolor}
\usepackage{tcolorbox}
\usepackage{caption}
\usepackage{hyperref}
\usepackage{float}
\geometry{margin=1in}

% Section styling
\titleformat{\section}
  {\normalfont\Large\bfseries\color{blue!60!black}}{\thesection}{1em}{}
\titleformat{\subsection}
  {\normalfont\large\bfseries\color{blue!40!black}}{\thesubsection}{1em}{}

\begin{document}

% -------- FIRST PAGE ONLY STYLING STARTS ---------
\begin{titlepage}
\begin{tcolorbox}[colback=blue!5!white, colframe=blue!80!black, width=\textwidth, arc=4mm, boxrule=1pt, title=Social Internship Report]

\begin{center}
    {\Huge \bfseries Social Internship Report}\\[2ex]
    {\LARGE NGO: Animal Welfare People, Dibrugarh}\\[2ex]
    \includegraphics[width=0.35\linewidth]{logo.png}\\[1ex]

    \vspace{2em}
    \textbf{\Large Submitted by:}

    \begin{itemize}
        \item Nayanmoni Khatoniar
        \item Rayon Kashyap
        \item Sanjeev Gogoi
        \item Princy Mohan
        \item Jubin Thakur
        \item Ankur Handique
        \item Arindram Borah
    \end{itemize}

    \vspace{2em}
    \textbf{Course:} B.Tech, Semester 2\\
    \textbf{Branch:} Chemical Engineering Department\\
    \textbf{College:} Golaghat Engineering College

    \vfill
    \textbf{Date:} \today
\end{center}

\end{tcolorbox}
\end{titlepage}
% -------- FIRST PAGE ONLY STYLING ENDS ---------

% Now your regular document starts
\newpage
\begin{document}




\maketitle
\thispagestyle{empty}
\newpage

% Table of contents
\tableofcontents
\newpage

% Sections
\section{Introduction}
\begin{tcolorbox}[colback=blue!5!white, colframe=blue!80!black, title=Introduction ]

    
This report outlines our experiences during a social internship with Animal Welfare People, an NGO based in Dibrugarh, Assam. They work for animal wellness and protection
\begin{center}
\includegraphics[width=0.7\textwidth]{11jp.jpg}
\captionof{figure}{Our first day at the NGO.}
\end{center}
\end{tcolorbox}



\section{Brief About What the NGO Does}
Animal Welfare People is a non-profit organization committed to the protection, care, and rehabilitation of animals in need. Their core activities include conducting rescue operations for injured, sick, or abandoned animals, and responding to cruelty cases reported by the public. The NGO plays a vital role in resolving such cruelty cases by providing immediate support and ensuring proper medical treatment and shelter for the affected animals.

In addition to rescue efforts, the organization regularly organizes food drives to feed stray animals across various localities, ensuring they receive basic nutrition. They also arrange awareness camps in public areas, educational institutions, and markets to spread knowledge about animal rights, responsible pet ownership, and the importance of humane treatment of animals.

Another key service offered by the NGO is its foster care initiative, where rescued animals are temporarily sheltered and looked after by volunteers or foster homes, free of charge, until a permanent adoption solution is found. These efforts are all aimed at creating a safer and more compassionate environment for animals within the community.

\begin{center}
\includegraphics[width=0.7\textwidth]{rescue1.jpg}
\captionof{figure}{Team at a rescue mission}
\end{center}

\begin{center}
\includegraphics[width=0.7\textwidth]{rescue2.jpg}
\captionof{figure}{Team at a rescue mission.}
\end{center}
\newpage


\section{Executive Committee and Member Details}

The NGO was founded by Late Vinit Bagaria, whose vision and dedication laid the foundation for Animal Welfare People. Since its inception, the organization has grown steadily, and today it has over 60+ active volunteers contributing to various initiatives and rescue operations.

The executive committee consists of 12 dedicated members who oversee the functioning and planning of the organization’s activities. The committee structure is as follows:

\begin{itemize}
    \item \textbf{President:} Gayatri Hazarika
    \item \textbf{General Secretary:} Subhajit Purkayastha
    \item \textbf{Vice President:} Sangeeta Gogoi
    \item \textbf{Advisor:} Anita Bagaria
    \item \textbf{Senior Rescuers:} Ayan Shah, Ashish Hussain, Manas Gogoi
\end{itemize}

Each member of the executive team plays a vital role in ensuring the smooth execution of the mission and day-to-day operations of the NGO.
\newpage


\section*{4. Activities by the NGO}



The NGO \textbf{Animal Welfare People}, founded in 2018, has been at the forefront of several humanitarian and animal welfare efforts across Assam. Over the years, the organization has carried out a wide range of activities, often driven by volunteers and community support.

\vspace{0.4cm}

\textbf{Key Activities:}

\begin{itemize}
    \item \textbf{Animal Rescue:} Rescuing injured and abandoned animals and ensuring they receive timely veterinary care.
    
    \item \textbf{Vaccination Drives:} Organizing free vaccination drives for street dogs and cats to prevent diseases like rabies.
    
    \item \textbf{Adoption Campaigns:} Promoting the adoption of rescued animals by connecting them with loving families.
    
    \item \textbf{Sterilization Programs:} Conducting sterilization surgeries to control the stray animal population in a humane way.
    
    \item \textbf{Food Drives:} Distributing home-cooked and packaged food items to street animals across various towns in Assam.
    
    \item \textbf{Awareness Programs:} Running public campaigns and school workshops to educate people about compassion toward animals, traffic safety, and the importance of coexisting peacefully with stray animals.
   \item \textbf{Cruelty Cases Solved:} The NGO has successfully intervened in numerous animal cruelty cases, working closely with local authorities to rescue abused animals and bring the perpetrators to justice. Their rapid response and legal awareness campaigns have helped reduce such incidents in many areas.
   \item \textbf{Fight Against Dog-Napping:} The organization actively combats dog-napping by tracking reported incidents, spreading awareness, and collaborating with law enforcement to rescue stolen pets and reunite them with their rightful families.
\item \textbf{Adoption Camp for Indigenous Breeds:} Special adoption drives are organized to promote the adoption of local and indigenous dog breeds, helping them find loving homes while encouraging people to support native animal welfare.


\end{itemize}
\newpage


\section{Our Experience}
\subsection*{Day 1: Briefing and Intro}
\item \textbf{Orientation and Introduction:} On the very first day of our internship, our team was warmly welcomed and introduced to the workings of the organization by the respected President, Gayatri Hazarika ma'am. She kindly introduced us to her adopted, specially-abled dogs — Hachy, Fusu, Ricky, and others — each with their own inspiring rescue story. 

Ma'am thoughtfully structured our internship schedule, ensuring that we received a holistic and organized experience throughout our time with the NGO. She shared valuable insights regarding animal safety laws, animal welfare rights, and the protocols followed during rescue missions. 

Additionally, she highlighted common social biases, particularly how indigenous dog breeds are often neglected or overlooked by potential adopters, stressing the importance of awareness and inclusive care. A significant part of the orientation included a discussion on the vital role of vaccinations — especially rabies and multi-vaccination drives — in ensuring both public safety and animal well-being.

Towards the end of the session, Gayatry ma'am also shared the emotional and inspiring backstory of the NGO, elaborating on how it was founded, the legacy of Late Vinit Bagaria, and how the organization evolved into a dedicated platform for animal protection and advocacy.

\begin{figure}
    \centering
    \includegraphics[width=0.5\linewidth]{in3.jpg}
    \caption{Huchy}
    \label{fig:enter-label}
\end{figure}
\newpage
\begin{figure}
    \centering
    \includegraphics[width=0.5\linewidth]{in2.jpg}
    \caption{Ricky}
    \label{fig:enter-label}
\end{figure}
\newpage


\subsection*{Day 2: Foster Home and Vet Visit}
\item \textbf{Foster Home and Veterinary Visit:} As part of our internship journey, we had the opportunity to visit a veterinary clinic where we became familiar with the professional environment and daily operations of a veterinary setup. The vet doctor patiently explained to us the common medical procedures, the importance of regular check-ups, and how timely intervention helps in preventing suffering among street animals. The visit gave us a clear understanding of the dedication and scientific approach required in the veterinary field. 

Following the clinic visit, we were taken to the foster home of Pronob sir — a compassionate and deeply kind-hearted individual who has selflessly turned his residence into a safe haven for rescued animals. Without seeking any financial benefit or external reward, he provides temporary shelter, food, and emotional care to abandoned and injured animals. His home radiated warmth and was a true example of how individual efforts can bring about meaningful change. Witnessing his selfless service was not only inspiring but also a reminder that empathy needs no institutional support to flourish.

\begin{figure}
    \centering
    \includegraphics[width=0.5\linewidth]{replace.jpg}
    \caption{photo with respected maam who adopted bon bon}
    \label{fig:enter-label}
\end{figure}
\subsection*{Day 3: Meeting Bon-Bon}
\item \textbf{Meeting Bon-Bon:} During one of our internship days, we were introduced to Bon-Bon, a rescued dog with a deeply emotional backstory. He was once a pet, but after being abandoned, he started roaming around the premises of Kanoi College. Due to his friendly nature, students and locals used to feed him, and he gradually became familiar with the environment.

However, a tragic incident occurred that changed everything. While playing with a college student, Bon-Bon playfully held the girl's hand in his mouth — an innocent gesture misunderstood by many. This act was misinterpreted as aggression, and rumors quickly spread that he was a mad dog. Fear took over, and people began throwing stones at him, treating him cruelly due to misinformation and panic.

Thankfully, he was rescued by the NGO just in time. Though Bon-Bon was severely traumatized by the incident, he is now on his journey of healing — both physically and emotionally. With time, care, and love, he is slowly learning to trust humans again. Meeting Bon-Bon was a powerful reminder of the importance of compassion, awareness, and second chances.
\begin{figure}[H]
    \centering
    \includegraphics[width=0.5\linewidth]{bon bon.jpg}
    \caption{bon bon }
    \label{fig:enter-label}
\end{figure}

\item \textbf{A Remarkable Act of Kindness:} On the same day, we witnessed another heartwarming moment that reinforced our respect for the NGO’s work. One of the kind-hearted members, referred to respectfully as “maam,” adopted three female puppies. It is widely known that female puppies are often left behind due to societal preferences and misconceptions. Her decision to adopt not just one, but three female puppies, was a bold and compassionate step toward challenging that bias and promoting equality in adoption. This act of kindness left a lasting impression on all of us.
\begin{figure}[H]
    \centering
    \includegraphics[width=0.5\linewidth]{pg4.jpg}
    \caption{the three female puppies }
    \label{fig:enter-label}
\end{figure}
\end{itemize}
\newpage

\item \textbf{Day 3: Food Drive:} On the third day of our internship, we conducted a heartfelt food drive for street dogs. Our team was divided into two groups for better coverage. One group of four members lovingly prepared home-cooked meals and personally fed the dogs with care and affection. The second group, consisting of three members, took charge of feeding dogs in another part of the town, offering them packets of Marie biscuits.

This activity not only helped feed many hungry strays but also allowed us to interact more closely with them, observe their behavior, and understand the importance of community-driven animal care. It was both an emotional and eye-opening experience that strengthened our commitment to the cause.

\begin{figure}[H]
    \centering
    \includegraphics[width=0.5\linewidth]{food1.jpg}
    \caption{home cooked meal distribution}
    \label{fig:enter-label}
\end{figure}

\begin{figure}[H]
    \centering
    \includegraphics[width=0.5\linewidth]{food2.jpg}
    \caption{home cooked meal distribution}
    \label{fig:enter-label}
\end{figure}

\begin{figure}[H]
    \centering
    \includegraphics[width=0.5\linewidth]{food3.jpg}
    \caption{biscuits distribution}
    \label{fig:enter-label}
\end{figure}

\begin{figure}[H]
    \centering
    \includegraphics[width=0.5\linewidth]{food4.jpg}
    \caption{biscuits distribution}
    \label{fig:enter-label}
\end{figure}

\begin{figure}[H]
    \centering
    \includegraphics[width=0.5\linewidth]{jubin.jpg}
    \caption{biscuits distribution}
    \label{fig:enter-label}
\end{figure}


\begin{figure}[H]
    \centering
    \includegraphics[width=0.5\linewidth]{food new.jpg}
    \caption{home cooked meal distribution}
    \label{fig:enter-label}
\end{figure}
\newpage





\subsection*{Day 4: Traffic Awareness Camp}
Participated in organizing a traffic awareness camp.

On the fourth day of our internship program, our team actively participated in a Traffic Awareness Campaign organized by the NGO. With a collective spirit of creativity and responsibility, we designed a total of 7 colorful and impactful posters using chart papers, sketch pens, and paints. Each poster carried a unique message related to traffic rules, pedestrian safety, responsible driving, and animal-friendly behavior on the roads. The goal was to convey crucial traffic guidelines in a visually engaging and easy-to-understand format for the general public.

After preparing the posters, we conducted a field campaign at two of the busiest junctions of Dibrugarh—Thana Chariali and Chowkidinghee Chariali. These locations were strategically chosen due to their heavy footfall and vehicular traffic. We displayed the posters prominently and interacted with commuters, passersby, and local shopkeepers to spread awareness. Many individuals paused to read the messages, take pictures, and even appreciate our efforts. Some shared their experiences with road safety, making the campaign not only interactive but also deeply informative.

This activity helped us understand the importance of public engagement in spreading awareness and how visual communication can influence behavior. Moreover, it emphasized the role NGOs can play in fostering civic responsibility and ensuring the safety of both humans and animals on the roads.

\begin{figure}[H]
    \centering
    \includegraphics[width=0.5\linewidth]{tt1.jpg}
    \caption{feature of our work in a media page}
    \label{fig:enter-label}
\end{figure}

\begin{figure}[H]
    \centering
    \includegraphics[width=0.5\linewidth]{tt2.jpg}
    \caption{feature of our work in a media page}
    \label{fig:enter-label}
\end{figure}

\begin{figure}[H]
    \centering
    \includegraphics[width=0.5\linewidth]{tt3.jpg}
    \caption{feature of our work in a media page}
    \label{fig:enter-label}
\end{figure}

\begin{figure}[H]
    \centering
    \includegraphics[width=0.5\linewidth]{tt5.jpg}
    \caption{feature of our work in a media page}
    \label{fig:enter-label}
\end{figure}

\begin{figure}[H]
    \centering
    \includegraphics[width=0.5\linewidth]{tt4.jpg}
    \caption{feature of our work in a media page}
    \label{fig:enter-label}
\end{figure}

\begin{figure}[H]
    \centering
    \includegraphics[width=0.5\linewidth]{ll.jpg}
    \caption{feature of our work in a media page}
    \label{fig:enter-label}
\end{figure}

\section{The Cat Case}
\section*{The Cat Case: Legal Action Against Animal Cruelty}

In early February 2025, a deeply disturbing incident came to light where an individual was seen torturing—and in certain aspects, even sexually harassing—a helpless cat. The video, which was uploaded and circulated through social media platforms, triggered outrage and concern among animal lovers and welfare organizations.

Our NGO, Animal Welfare People, took immediate cognizance of the matter. After verifying the authenticity of the footage and gathering preliminary evidence, we acted promptly and filed a legal case against the accused individual. The act was in direct violation of the Prevention of Cruelty to Animals Act, and due to the severity of the abuse, legal proceedings commenced without delay.

From February 2025 to July 2025, the accused was regularly summoned and made to appear in court as the trial progressed. Our team actively monitored every step of the judicial process, ensuring that no loophole was left unattended. Through consistent legal follow-up and community pressure, the case remained alive in the public consciousness.

Eventually, the case concluded with a conditional withdrawal. The court issued a strict warning to the accused: should he, at any point in the future, repeat even a minor act of cruelty or neglect towards any cat—or any animal—he would be subject to immediate arrest and would face direct imprisonment without further warning.

This case stands as a testament to our NGO's commitment to the protection of animal rights and our zero-tolerance approach toward acts of cruelty and abuse.

\newpage

\section{Our Own Testimonials}

\section*{Our Own Testimonials}

\textbf{Team Testimonial:} 

Participating in this internship under the Animal Welfare People NGO has been nothing short of transformative for all of us. Each of us walked in with our own perceptions and experiences about animals, society, and service. But over the days, through food drives, adoption and awareness campaigns, street activities, and even intense legal follow-ups like “The Cat Case,” we found ourselves growing—not just as individuals, but as a united force.

We did not just distribute food to hungry street dogs—we learned to overcome our fear, hesitation, and the unknown. We didn’t just hold a traffic awareness campaign—we realized the power of creativity in advocacy, how a small message on chart paper could influence a passerby and make roads safer not only for people, but for animals too. We didn’t just visit shelters—we listened, observed, and connected with those who give their lives to voiceless beings.

The biggest realization, however, came from seeing how impactful small acts of love and effort can be when done with intention. Preparing food at home, making posters together, splitting into groups to cover more ground—these small choices gave us a sense of purpose that no classroom ever taught.

And above all, we discovered the strength in each other. Every member of this team brought something unique—be it compassion, leadership, artistic expression, legal awareness, or silent dedication. Together, we created an experience that we will always carry forward with pride, empathy, and a deeper sense of responsibility toward the world around us.

This internship didn’t just help animals. It helped shape us into more responsible, aware, and emotionally intelligent individuals—and that’s a gift we’ll cherish long after the last report is written.

\begin{figure}[H]
    \centering
    \includegraphics[width=0.5\linewidth]{testi3.jpg}
    \caption{selected testimonials from our teammates}
    \label{fig:enter-label}
\end{figure}

\begin{figure}[H]
    \centering
    \includegraphics[width=0.5\linewidth]{testi1.jpg}
    \caption{selected testimonials from our teammates}
    \label{fig:enter-label}
\end{figure}

\begin{figure}[H]
    \centering
    \includegraphics[width=0.5\linewidth]{testi2.jpg}
    \caption{selected testimonials from our teammates}
    \label{fig:enter-label}
\end{figure}
\newpage

\section{Special Acknowledgements}
\section*{Special Acknowledgements}

We extend our heartfelt gratitude to \textbf{Gayatri Ma’am} for graciously giving us the opportunity to be a part of this deeply enriching internship experience under the guidance of the Animal Welfare People NGO. 

Your faith in our potential and your kind support allowed us to engage meaningfully in the field of animal welfare, something that has left a lasting impact on each of us. From the very beginning, your encouragement gave us the confidence to step out of our comfort zones and take up responsibilities that have shaped our sense of compassion, leadership, and social awareness.

It is through your generosity and thoughtful mentorship that we were able to understand the deeper value of community service—not just as a task or obligation, but as a genuine act of empathy and purpose. Each drive we conducted, every interaction we had, and all the challenges we faced became learning experiences that we now carry forward with pride and gratitude.

Thank you, Ma’am, for making this journey possible, for guiding us with such grace, and for being the bridge that connected us with a cause greater than ourselves. Your belief in us has meant the world, and we are truly thankful for having had the chance to learn and grow through this opportunity.



\section{Photo Gallery}
\begin{figure}[H]
    \centering
    \includegraphics[width=0.5\linewidth]{pg1.jpg}
    \caption{memories with the ngo}
    \label{fig:enter-label}
\end{figure}

\begin{figure}[H]
    \centering
    \includegraphics[width=0.5\linewidth]{pg2.jpg}
    \caption{memories with the ngo}
    \label{fig:enter-label}
\end{figure}
\end{document}
